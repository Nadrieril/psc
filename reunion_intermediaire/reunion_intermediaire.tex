\documentclass{beamer}
\usepackage{pgfpages}
\usepackage[francais]{babel}
\usepackage[T1]{fontenc}
\usepackage[utf8]{inputenc}
\usepackage{amsmath,amsthm}
\usepackage{amsfonts,amssymb}

%%%%%%%%%%%%%%%%%%%%%%%%%%%%%%%%%%%%%%%%%%%%%%%%%%%%%%%%%%%%%%%%%%%%%%%%%%%%%%%
%THEMES DU DOCUMENT
%%%%%%%%%%%%%%%%%%%%%%%%%%%%%%%%%%%%%%%%%%%%%%%%%%%%%%%%%%%%%%%%%%%%%%%%%%%%%%%

%– th`emes globaux~: albatross, beetle, crane, fly, seagull.
%– th`emes internes~: lily  orchid, rose.
%– th`emes externes~: whale, seahorse, dolphin
%JuanLesPins Malmoe PaloAlto Berlin Boadilla Copenhagen Hannover Goettingen Montpellier Rochester
%Madrid Antibes Singapore Szeged Warsaw, thème par défaut Ilmenau Luebeck AnnArbor CambridgeUS Dresden

%themes de polices~: professionalfonts serif structurebold structureitalicserif structuresmallcapsserif


\usetheme{Warsaw}
%\usecolortheme{dolphin}
\usefonttheme{structurebold}
%\useinnertheme{nom du theme interne}
%\useoutertheme{split}

%%%%%%%%%%%%%%%%%%%%%%%%%%%%%%%%%%%%%%%%%%%%%%%%%%%%%%%%%%%%%%%%%%%%%%%%%%%%%%%%%%%%%%%%
%AUTRES COMMANDES PRÉLIMINAIRES
%%%%%%%%%%%%%%%%%%%%%%%%%%%%%%%%%%%%%%%%%%%%%%%%%%%%%%%%%%%%%%%%%%%%%%%%%%%%%%%%%%%%%%%%%

\AtBeginSection[]{% génère une tableofcontents au début de chaque section
  \begin{frame}
  \frametitle{Sommaire}
	\tableofcontents[sectionstyle = show/shaded,subsectionstyle = show/show/shaded]
  \end{frame}
}

\setbeamertemplate{itemize item}[triangle]
\setbeamertemplate{enumerate item}[square]
\setbeamertemplate{blocks} [rounded] [shadow=true]

\setbeameroption{show notes}%%ou hide notes
%%\setbeameroption{show notes on second screen=right}


\title{Réunion intermédiaire}
\subtitle{PSC INF02~: Synthétiseur automatique de documents}
\author{}
\institute{École polytechnique}
\date{}

\logo{\includegraphics[height=0.8cm]{logohori.eps}}

%%%%%%%%%%%%%%%%%%%%%%%%%%%%%%%%%%%%%%%%%%%%%%%%%%%%%%%%%%%%%%%%%%%%%%%%%%%%%%%%%%%
%DEBUT DU DOCUMENT
%%%%%%%%%%%%%%%%%%%%%%%%%%%%%%%%%%%%%%%%%%%%%%%%%%%%%%%%%%%%%%%%%%%%%%%%%%%%%%%%%%%

\begin{document}




%%page de titre
\begin{frame}
  \titlepage{}
\end{frame}

%% sommaire %%
% \begin{frame}
% \frametitle{Sommaire}
% \tableofcontents
% \end{frame}


%%%%%%%%%%%%%%%%%%%%%%%%%%%%%%%%%%%%%%%%%%%%%%%%%%%%%%%%%%%%%%%%%%%%%%%%%%%%
%VRAI DEBUT DU DOCUMENT
%%%%%%%%%%%%%%%%%%%%%%%%%%%%%%%%%%%%%%%%%%%%%%%%%%%%%%%%%%%%%%%%%%%%%%%%%%%%

\section{Point d'avancement}

\begin{frame}
\frametitle{Point d'avancement}
	\begin{block}
		\begin{itemize}
			\item Traitement syntaxique offert par systran.
			\item Réseau de concepts opérationnel
			\item TF-IDF opérationnel
			\item Workspace en cours de développement
		\end{itemize}
	\end{block}
\end{frame}

\begin{frame}
\frametitle{Changements remarquables}
	\begin{block}
		\begin{itemize}
			\item Les workers ne semblent plus indispensables à notre travail
			\item Nous nous orientons désormais vers des résumés très synthétiques (une ou deux phrases au plus)
		\end{itemize}
	\end{block}
\end{frame}

\section{Ce qui est fait}
% Disons, une sous section par sujet ? Avec une ou deux diapo par sujet, aussi.

\section{Prochaines étapes}
% Cf supra.
% En particulier : le workspace, l'importance conceptuelle.


\section{Synthèse}
% Un petit résumé, ça fait pas de mal !

\end{document}

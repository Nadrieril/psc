\documentclass{beamer}
\usepackage{pgfpages}
\usepackage[frenchb]{babel}
\usepackage[T1]{fontenc}
\usepackage[utf8]{inputenc}
\usepackage{amsmath,amsthm}
\usepackage{amsfonts,amssymb}

\usetheme{Hannover}
\usecolortheme{seahorse}
\usefonttheme{structurebold}
\setbeamercovered{dynamic}

\AtBeginSection[]{%
  \begin{frame}
    \frametitle{Sommaire}
    \tableofcontents[sectionstyle = show/shaded,subsectionstyle = show/show/shaded]
  \end{frame}
}

\usepackage{autoTitle}
\usepackage{booktabs}

\setbeamertemplate{itemize item}[triangle]
\setbeamertemplate{enumerate item}[square]

\title{Réunion intermédiaire}
\subtitle{PSC INF02~: Synthétiseur automatique de documents}
\author{}
\institute{École polytechnique}
\date{}

\logo{\includegraphics[height=0.8cm]{logohori.eps}}

\begin{document}
\beamertemplatenavigationsymbolsempty{}
\beamerdefaultoverlayspecification{<+->}
\begin{frame}
  \titlepage{}
\end{frame}

\section{Évolution du projet}
\subsection{Avancement général}

\begin{frame}
  \begin{block}{État actuel}
    \begin{itemize}
      \item Traitement syntaxique offert par systran.
      \item Réseau de concepts opérationnel
      \item TF-IDF opérationnel
      \item Workspace en cours de développement
    \end{itemize}
  \end{block}
\end{frame}

\subsection{Réseau de concepts}
% Schrotty !
\begin{frame}

\end{frame}

\subsection{Élaboration du workspace}
\begin{frame}
  \begin{block}{Ajustements de direction}
    \begin{itemize}
      \item Les workers ne semblent plus indispensables à notre travail
      \item Nous nous orientons désormais vers des résumés très synthétiques (une ou deux phrases au plus)
    \end{itemize}
  \end{block}

  \begin{block}{Structure du workspace}
   \begin{itemize}
     \item Projection du texte dans un espace plus abstrait
     \item L'importance d'un concept dans le workspace correspond à l'importance sémantique dans le texte
   \end{itemize} 
  \end{block}
\end{frame}

\section{Prochaines étapes}
% Cf supra.
% En particulier : le workspace, l'importance conceptuelle.

\section{Synthèse}
% Un petit résumé, ça fait pas de mal !

\end{document}

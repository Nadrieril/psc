\documentclass{article}   
\usepackage[utf8]{inputenc}     
\usepackage[T1]{fontenc}      
\usepackage[french]{babel}
\usepackage[pdftex]{graphicx}
\usepackage{graphics}

\usepackage{amsmath}
\usepackage{amssymb}
\usepackage{amsfonts}
\usepackage{graphics}
\usepackage{fancybox}		
\usepackage{fancyhdr}
\usepackage{makeidx}              %% permet de générer un index automatiquement

\usepackage[top=3cm, bottom=4cm, left=2cm, right=2cm,headsep = 0.5cm,headheight = 1cm, footskip = 1cm,marginparsep = 0cm,marginparwidth=0cm]{geometry} %%pour aller vite

\pagestyle{fancy}
\renewcommand\headrulewidth{1pt}
\fancyhead[L]{Réunion}
\fancyhead[R]{Document interne}
\fancyfoot[L]{PSC du mythe 2014}
\fancyfoot[R]{}

\title{Réunion PSC}
\author{}

\bibliographystyle{plain}	  %% le style utilisé pour créer la bibliographie
\begin{document}                 
\maketitle
    
\section{Informations}

Réunion interne effectuée le 17 novembre 2014 dans le casert de Guillaume.

But de la réunion : fixer les nouveaux objectifs.

Durée/productivité : $\infty$.

\section{Bilan}

\subsection*{Urgent}

\begin{itemize}
 \item Finir implémentation du TFID
 \item Trouver la structure du réseau
 \item A quoi ressemble le workspace ?


\subsection*{Variations autour du réseau de concept}

\begin{itemize}
 \item Première approche par co-occurrence des mots.
 \item Problème de la symétrie du graphe : Graphe orienté ou pas ? bouleau => arbre mais pas réciproquement. Cependant "arbre" est relié avec beaucoup d'espèces d'arbre ce qui peut donner une symétrie...
\end{itemize}


\subsection*{Qui fait quoi ?}


\begin{itemize}
 \item Théophane finit le parseur.
 \item Antonin-san travaille sur le parseur et la grammaire.
 \item Guillaume retravaille le code de Schrotty.
 \item Sarah réfléchit.
 \item Zhixing finit le TFID ?
 \item Schrotty !
\end{itemize}


\section{Conclusion}

Réunion peu productive où nous sommes parvenus à l'idée qu'avoir un réseau de concept, c'est bien.

\end{document}

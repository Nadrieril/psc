\documentclass{article}           %% ceci est un commentaire (apres le caractere %)
\usepackage[utf8]{inputenc}   
\usepackage[T1]{fontenc}          %% permet d'utiliser les caractères accentués
\usepackage[french]{babel}
\usepackage[pdftex]{graphicx}
\usepackage{graphics}

\usepackage{fancybox}		   %% package utiliser pour avoir un encadré 3D des images
\usepackage{fancyhdr}
\usepackage{makeidx}              %% permet de générer un index automatiquement
\usepackage[style=numeric,backend=bibtex]{biblatex}				%% Utilisé pour la biblio
\usepackage{hyperref}


%\pagestyle{fancy}
%\renewcommand\headrulewidth{1pt}
%\fancyhead[L]{Rapport interm\'ediaire}
%\fancyhead[R]{}
%\fancyfoot[L]{}
%\fancyfoot[R]{}

\newtheorem{definition}{D\'efinition}
\addto\captionsfrench{\renewcommand{\contentsname}{Sommaire}}


\title{Projet Scientifique Collectif X2013 \\INF02~: Synthétiseur automatique de documents}     %% \title est une macro, entre { } figure son premier argument

\author{Rapport intermédiaire}
\date{21 janvier 2015}
%mettre les accents comme \c{c}a, sinon risque de plantage
\index{R\'eseau S\'emantique}
\index{Analyse Symbolique}
\index{Analyse S\'emantique}
\index{R\'eseau de Concepts}

\makeindex
\addbibresource{biblio.bib}
\begin{document}                  %% signale le début du document

\maketitle  

\newpage

\tableofcontents			
\newpage

\section{Rappel du projet}

\subsection{Notre groupe}
\begin{itemize}
 \item Fernandes-Pinto-Fachada Sarah, \textbf{8\textsuperscript{e}} compagnie, section \textbf{\'equitation};
 \item Schrottenloher Andr\'e, \textbf{8\textsuperscript{e}} compagnie, section \textbf{escrime};
 \item Angibault Antonin, \textbf{8\textsuperscript{e}} compagnie, section \textbf{escrime};
 \item Hufschmitt Th\'eophane, \textbf{8\textsuperscript{e}} compagnie, section \textbf{escrime};
 \item Cao Zhixing, \textbf{9\textsuperscript{e}} compagnie, section \textbf{escalade};
 \item Boisseau Guillaume, \textbf{6\textsuperscript{e}} compagnie, section \textbf{natation};
\end{itemize}

\subsection{Notre sujet}
Analyse s\'emantique\index{Analyse S\'emantique}

\section{\'Etat d'avancement~: r\'ecup\'eration de donn\'ees}

\section{Le r\'eseau de concepts}


\subsection{Structure}


\begin{definition}[Réseau de concepts]
  Le réseau de concepts\index{R\'eseau de Concepts} est la ``mémoire à long terme'' de notre programme.

Ce réseau est au carrefour entre les réseaux sémantiques et les réseaux de neurones. Les n\oe{}uds du réseau ne représentent pas des concepts à proprement parler~; en revanche, les concepts sont vus comme regroupant des n\oe{}uds fortement interconnectés.
\end{definition}

Le RC est constitué de n\oe{}uds. Chaque n\oe{}ud comporte~:
\begin{itemize}
  \item Une étiquette (mot)~;
 \item Une importance conceptuelle (IC) ou profondeur conceptuelle~;
 \item Une activation (A) initialement nulle~;
 \item Un certain nombre de liens à d'autres n\oe{}uds.
\end{itemize}

Chaque lien comporte~:
\begin{itemize}
 \item Une proximité conceptuelle (P)~;
 \item Une étiquette que nous reprenons sur le modèle de conceptnet
\end{itemize}

\subsection{Construction}

\subsubsection{Provenance des données~?}

Le but est donc de créer un réseau de concepts de base, que nous puissions utiliser comme base de connaissances dans le domaine considéré (ici le sport, mais nous pourrions avoir sélectionné un autre domaine).

Nous utilisons deux types de sources~:
\begin{itemize}
 \item Des articles de sport (données brutes) qui contiennent entre autres ``ce qu'il faut savoir''~;
 \item Des données sémantiques libres.
\end{itemize}

\subsubsection{Construction du réseau, première étape}

Après la lecture d'un grand nombre d'articles provenant de sources diverses (flux rss et sites d'information consacrés au sport), ceux-ci sont analysés pour en retirer un certain nombre de concepts sous forme normale. Cette opération se traduit dans la pratique en~:
\begin{itemize}
 \item Découpage du texte en phrases et en mots (tokens)~;
 \item Utilisation du part-of-speech tagger par défaut de nltk~;
 \item Récupération des noms propres et mise à part de ceux-ci~;
 \item Suppression des conjonctions de coordination, pronoms et autres mots qui ne sont pas de véritables concepts~;
 \item Il reste alors des adjectifs, adverbes, noms et verbes. Nous utilisons morphy de wordnet, ainsi qu'un morceau de code déjà utilisé pour construire conceptnet5 à partir de wordnet (voir partie suivante), pour transformer chaque terme en sa forme normale. Les noms sont mis au singulier, les verbes à l'infinitif\ldots{}~;
 \item Les termes restants sont alors tous les concepts auxquels le texte fait appel. Nous savons désormais que ces concepts devront apparaître dans le réseau, sauf exception (erreur à l'une des étapes)~;
 \item De manière générale, un concept qui apparaît au moins deux fois dans l'ensemble du corpus considéré peut être considéré comme valide (il y a toujours un risque d'erreur, mais il a été minimisé).
\end{itemize}


\subsubsection{WordNet}

WordNet est un dictionnaire complet contenant une grande quantité d'informations sur les mots.

Nous l'utilisons à travers son interface nltk, pour récupérer des informations sur les mots et les mettre sous forme normale.


\subsubsection{Conceptnet5}

ConceptNet est un projet libre de réseau sémantique représentant des connaissances usuelles, aussi bien de la vie de tous les jours, culturelles, scientifiques. Il fait partie du Commonsense Computing Initiative qui relie différents laboratoires, dont le MIT Media Lab, et entreprises.

Il est important de noter que conceptnet5 est généré à partir de données brutes. Conceptnet5 est relié à DBPedia, une grande partie de ses connaissances provient de Wiktionary, une partie de WordNet.
``The knowledge in ConceptNet is collected from a variety of resources, including crowd-sourced resources (such as Wiktionary and Open Mind Common Sense), games with a purpose (such as Verbosity and nadya.jp), and expert-created resources (such as WordNet and JMDict).''

Page web du projet~:
http://conceptnet5.media.mit.edu/

Le code python permettant de générer conceptnet5 à partir de données brutes est sur la page~:
https://github.com/commonsense/conceptnet5

\subsubsection{Structure de conceptnet5}

Conceptnet5 est un graphe sémantique de 
les n\oe{}uds sont des mots et de courtes phrases, dans un certain nombre de langages possibles. Les arêtes qui relient ces n\oe{}uds expriment une connaissance, elles expriment chacune une relation particulière.

Une arête possède aussi une source (d'où provient l'information) et un poids en fonction de cette source, selon l'importance de l'arête.

Une relation concept-arête-concept exprime une assertion. Une même assertion peut être exprimée de différentes manières.

Une assertion peut elle-même être utilisée comme un n\oe{}ud ou comme une arête (on peut avoir des assertions d'assertions).

Les relations valables dans tout langage, pour ConceptNet5, sont par exemple~: 
\begin{itemize}
 \item RelatedTo
 \item IsA
 \item PartOf
 \item MemberOf
 \item HasA
 \item UsedFor
 \item CapableOf
 \item AtLocation
 \item Causes
 \item HasSubevent
 \item HasFirstSubevent
 \item HasLastSubevent
 \item HasPrerequisite
 \item HasProperty
 \item MotivatedByGoal
 \item ObstructedBy
 \item Desires
 \item CreatedBy
 \item Synonym
 \item Antonym
 \item DerivedFrom
 \item TranslationOf
 \item DefinedAs
\end{itemize}

https://github.com/commonsense/conceptnet5/wiki/Relations

La correspondance entre relations de WordNet et de Conceptnet est par exemple~:
\begin{itemize}
 \item `attribute': `Attribute',
 \item `causes': `Causes',
 \item `classifiedByRegion': `HasContext',
 \item `classifiedByUsage': `HasContext',
 \item `classifiedByTopic': `HasContext',
 \item `entails': `Entails',
 \item `hyponymOf': `IsA',
 \item `instanceOf': `InstanceOf',
  \item  `memberMeronymOf': `MemberOf',
  \item  `partMeronymOf': `PartOf',
  \item  `sameVerbGroupAs': `SimilarTo',
  \item  `similarTo': `SimilarTo',
  \item  `substanceMeronymOf': `MadeOf',
  \item  `antonymOf': `Antonym',
  \item  `derivationallyRelated': `DerivedFrom',
  \item  `pertainsTo': `PertainsTo',
  \item  `seeAlso': `RelatedTo',
\end{itemize}

Nous avons décidé après coup de conserver ces relations dans le réseau de concepts, et d'en utiliser un sous-ensemble.

\subsubsection{Détail de l'API}

Nous pouvons faire un certain nombre de requêtes à Conceptnet5. Il existe trois types différents de requêtes~:
\begin{itemize}
 \item Lookup
 \item Association
 \item Search
\end{itemize}
Association permet de calculer la proximité entre deux concepts, ou de récupérer une liste de concepts proches d'un concept donné.

Search permet de récupérer une liste d'arêtes (edges) entre concepts, selon les paramètres spécifiés (le plus souvent, on impose le concept de départ start ou d'arrivée end). 

Lookup permet d'analyser un concept (on aura par exemple accès à des listes d'arêtes dans lequel il intervient).


\subsubsection{Utilisation de Conceptnet5}

Nous utilisons en majorité Association et Search.
Étant donné un concept dont nous savons qu'il doit être étendu, nous utilisons conceptnet5 pour créer de nouveaux concepts au sein du réseau, et pour ajoutr de nouveaux liens. Une première requête de type Search permet d'avoir accès à un certain nombre de liens vers d'autres concepts, qui sont ajoutés (on ne récupère que les plus pertinents). Une requête de type Association permet de créer de nouveaux liens vers d'autres concepts similaires.

Nous ajoutons alors une arête SimilarTo.



    
\subsubsection{Freebase}

Freebase est une immense base de données sémantiques qui contient beaucoup d'informations sur des noms propres notamment. Le projet, repris par google mais sous une license qui laisse les données libres d'accès, de téléchargement et d'utilisation, a lui aussi une API, un peu plus complexe que celle de conceptnet5. Freebase permet notamment de récuperer des informations sur une personne à partir de son nom, et à peu de frais, de vérifier que cette personne existe bel et bien.

    
    

\section{Traitement du r\'eseau}

\section{Suites envisag\'ees}

\section{R\'ef\'erences bibliographiques}

\nocite{*}
\printbibliography{}



\appendix

\printindex


\end{document}

\documentclass{beamer}
\usepackage{pgfpages}
\usepackage[francais]{babel}
\usepackage[T1]{fontenc}
\usepackage[utf8]{inputenc}
\usepackage{amsmath,amsthm}
\usepackage{amsfonts,amssymb}

%%%%%%%%%%%%%%%%%%%%%%%%%%%%%%%%%%%%%%%%%%%%%%%%%%%%%%%%%%%%%%%%%%%%%%%%%%%%%%%
%THEMES DU DOCUMENT
%%%%%%%%%%%%%%%%%%%%%%%%%%%%%%%%%%%%%%%%%%%%%%%%%%%%%%%%%%%%%%%%%%%%%%%%%%%%%%%

%– th`emes globaux~: albatross, beetle, crane, fly, seagull.
%– th`emes internes~: lily  orchid, rose.
%– th`emes externes~: whale, seahorse, dolphin
%JuanLesPins Malmoe PaloAlto Berlin Boadilla Copenhagen Hannover Goettingen Montpellier Rochester
%Madrid Antibes Singapore Szeged Warsaw, thème par défaut Ilmenau Luebeck AnnArbor CambridgeUS Dresden

%themes de polices~: professionalfonts serif structurebold structureitalicserif structuresmallcapsserif


\usetheme{Warsaw}
%\usecolortheme{dolphin}
\usefonttheme{structurebold}
%\useinnertheme{nom du theme interne}
%\useoutertheme{split}

%%%%%%%%%%%%%%%%%%%%%%%%%%%%%%%%%%%%%%%%%%%%%%%%%%%%%%%%%%%%%%%%%%%%%%%%%%%%%%%%%%%%%%%%
%AUTRES COMMANDES PRÉLIMINAIRES
%%%%%%%%%%%%%%%%%%%%%%%%%%%%%%%%%%%%%%%%%%%%%%%%%%%%%%%%%%%%%%%%%%%%%%%%%%%%%%%%%%%%%%%%%

\AtBeginSection[]{% génère une tableofcontents au début de chaque section
  \begin{frame}
  \frametitle{Sommaire}
	\tableofcontents[sectionstyle = show/shaded,subsectionstyle = show/show/shaded]
  \end{frame}
}

\setbeamertemplate{itemize item}[triangle]
\setbeamertemplate{enumerate item}[square]
\setbeamertemplate{blocks} [rounded] [shadow=true]

\setbeameroption{show notes}%%ou hide notes
%%\setbeameroption{show notes on second screen=right}


\title{Réunion de lancement}
\subtitle{PSC INF02~: Synthétiseur automatique de documents}
\author{}
\institute{École polytechnique}
\date{}

\logo{\includegraphics[height=0.8cm]{logohori.eps}}

%%%%%%%%%%%%%%%%%%%%%%%%%%%%%%%%%%%%%%%%%%%%%%%%%%%%%%%%%%%%%%%%%%%%%%%%%%%%%%%%%%%
%DEBUT DU DOCUMENT
%%%%%%%%%%%%%%%%%%%%%%%%%%%%%%%%%%%%%%%%%%%%%%%%%%%%%%%%%%%%%%%%%%%%%%%%%%%%%%%%%%%

\begin{document}




%%page de titre
\begin{frame}
  \titlepage{}

\end{frame}
%% sommaire %%
\begin{frame}
\frametitle{Sommaire}
\tableofcontents
\end{frame}


%%%%%%%%%%%%%%%%%%%%%%%%%%%%%%%%%%%%%%%%%%%%%%%%%%%%%%%%%%%%%%%%%%%%%%%%%%%%
%VRAI DEBUT DU DOCUMENT
%%%%%%%%%%%%%%%%%%%%%%%%%%%%%%%%%%%%%%%%%%%%%%%%%%%%%%%%%%%%%%%%%%%%%%%%%%%%

\section{Introduction}

\begin{frame}%[allowframebreaks=0.8]
\frametitle{Objectifs du projet}
%%à l'ère du big data, une grande partie de ces données sont textuelles

\begin{block}{Des problèmes nombreux}
\begin{itemize}
 \item De grandes quantités de données textuelles
 \item Une grammaire complexe (langage naturel)
 \item Beaucoup de prérequis de bon sens ou de culture générale
\end{itemize}

\end{block}


\begin{block}{Plusieurs solutions envisageables}
\begin{itemize}
\item «~Comprendre~» l'information du texte et la retranscrire~: \textbf{résumé abstractif}
\item Analyser le texte de façon statistique et en extraire des phrases~: \textbf{résumé extractif}
\end{itemize}
\end{block}


\end{frame}


\begin{frame}
\frametitle{Notre projet~: Synthétiseur automatique de documents}

Nous voulons résumer un texte en langage naturel, de façon abstractive.

Le résumé final ne sera pas une simple copie des phrases importantes du texte.

\end{frame}

\section{Structure du projet}

\subsection{Le projet}
\begin{frame}
\frametitle{Le synthétiseur en six étapes}

\begin{columns}[t]

\begin{column}{6cm}
\begin{enumerate}
 \item Analyse syntaxique
% ie la grammaire. Nos exigences devraient entrer dans le champ des possibilités des outils libres déjà existants
 \item Repérage des entités.
 %les entités ne sont rien d'autre que des concepts importants
 \item Connaissances des liens entre entités.
 %liens qui sont donnés par FB~: Zlatan est un joueur de foot par exemple
 \item Analyse des concepts.
 %il s'agit de relier tous les mots à des concepts
 \item Approfondissement des liens entre les concepts.
 %explicité dans la suite
 \item Génération de texte simple
\end{enumerate}
\end{column}
%à droite, ce que ça donne aux différents stades

\begin{column}{6cm}
\begin{enumerate}
\item Sujet, Verbe, Objet\ldots{}
\item Zinedine Zidane\ldots{}
\item Zidane est un joueur de football\ldots{}
\item Le football est un sport\ldots{}
\item Zinedine Zidane est donc un sportif.
\item Sujets-Actions-Objets-Compléments
\end{enumerate}

\end{column}
\end{columns}

\end{frame}

\begin{frame}
\frametitle{Déroulement du projet}

\begin{enumerate}
\item Étude statistique
\item Étude syntaxique et linguistique
\item Étude «~intelligente~»
\item Génération et étude de la qualité du résumé
\end{enumerate}

\end{frame}

\subsection{Organisation du travail}

\begin{frame}
\frametitle{Méthodes de travail}
	\begin{itemize}
		\item Des réunions hebdomadaires, pour~:
		\begin{itemize}
			\item Faire le point sur le travail effectué dans la semaine;
			\item Rendre compte et résoudre les difficultés rencontrées;
			\item Définir les axes prioritaires pour la semaine à venir.
		\end{itemize}
		\item Un répertoire partagé (git) pour une gestion pratique des fichiers.
	\end{itemize}
\end{frame}

\begin{frame}
\frametitle{Moyens utilisés}

De nombreux outils \textit{open-source} ou libres, notamment~:
\begin{itemize}
 \item Freebase
 \item WordNet (Princeton)
 \item nltk (librairie python)
\end{itemize}
\end{frame}



\section{Étude statistique}
\begin{frame}
\frametitle{TF-IDF(Term Frequency-Inverse Document Frequency)}
\begin{itemize}
  \item TF:la fréqence d'un terme est le nombre d'occurrences de ce terme dans le document considéré.
  \item IDF:la fréquence inverse de document est une mesure de l'importance du terme dans l'ensemble du corpus.
$IDF_{i}=log \frac{|D|}{|d_{j}:t_{i}\in d_{j}|}$
  \item TF-IDF:le multiple de TF et IDF
  \textbf{ex:le poids de mot i sur le document j}
   $TF-IDF_{i,j} = TF_{i,j} \times IDF_{i}$
\end{itemize}
\end{frame}

\begin{frame}
\frametitle{TF-IDF et la similarité de document}
TF-IDF est fréquemment utilisé pour construire un modèle d'espace vectoriel:
%%un graphe que je n'ai pas réussi à ajouter...
Ici, on considère deux documents avec deux mots-clés: "gossip" et "jealous"
\end{frame}

\begin{frame}
\frametitle{TF-IDF et la résumé automatique}
\begin{itemize}
  \item calculer le TF de chaque mot
  \item enlever les mots vides
  \item calculer le TF-IDF et trier les mots valeur
  \item calculer les valeurs des phrases à partir des mots-clés
  \item trier les phrases
\end{itemize}
\end{frame}

\begin{frame}
\frametitle{Problème de ce modèle}
\begin{itemize}
  \item les facteurs considérés sont trop simples
  \item répétition de l'information
  \item risque de choisir des phrases trop longues
\end{itemize}
À venir: chercher d'autres indices et compléter le modèle
\end{frame}

\section{Structure de l'analyseur}%Guillaume
    \begin{frame}
        \frametitle{Exemple - Fil rouge}
        Phrase à analyser:\newline{}
        "Wayne Rooney celebrated his appointment as England captain with the winner against Norway on a night when
        fans turned their back on the team following their dismal World Cup exit."
    \end{frame}

    \subsection{Réseau de concepts}
        \begin{frame}
        \frametitle{Réseau de concepts}
            \begin{itemize}
                \item Graphe de concepts (objets, actions ou idées plus abstraites) reliés entre eux;
                \item Deux concepts sont proches s'il est pertinent de penser au premier après avoir rencontré le second;
                \item Modélise une certaine compréhension du monde;
                \item Capable d'apprendre et d'évoluer à partir des résultat d'une analyse de texte.
            \end{itemize}
        \end{frame}

        \begin{frame}
        \frametitle{Exemple}
            \includegraphics[height=0.8\textheight]{RC/figures/RCetape0.png}
        \end{frame}

        \begin{frame}
        \frametitle{Concept clé: l'activation}
            \includegraphics[height=0.8\textheight]{RC/figures/RCetape30.png}
        \end{frame}

    \subsection{Workspace}
        \begin{frame}
        \frametitle{Workspace}
            \begin{itemize}
                \item Zone de travail qui contient les \textit{instances} des concepts rencontrés et des structures en cours de construction;
                \item Modélise la progression de la compréhension du texte;
            \end{itemize}
        \end{frame}

        \begin{frame}
        \frametitle{Exemple}
        \end{frame}

    \subsection{Workers}
        \begin{frame}
        \frametitle{Workers}
            \begin{itemize}
                \item Ensemble de méthodes permettant d'extraire de l'information du texte et du réseau de concepts;
                \item Gèrent l'activation du réseau de concepts, la création/modification de structures du Workspace, etc.;
                \item Sont executés au fur et à mesure de l'analyse et gèrent son déroulement.
            \end{itemize}
        \end{frame}

        \begin{frame}
        \frametitle{Exemple}
            \begin{itemize}
                \item Workers gérant la lecture mot-à-mot du texte;
                \item Workers capables d'identifier les références à un même objet;
                \item Workers comprenant l'importance des mots selon l'ordre du texte, les paragraphes, le titre;
                \item Workers capables d'extraire les structures les plus importantes.
            \end{itemize}
        \end{frame}

    \subsection{Synthèse}
        \begin{frame}
        \frametitle{Synthèse}
            \begin{itemize}
                \item Le réseau de concepts contient des connaissances \textit{a priori} sur le monde; il n'est pas modifié pendant l'analyse;
                \item Le workspace est la zone de travail qui contient les \textit{instances} des concepts rencontrés ainsi que des structures;
                \item Les workers sont différentes méthodes de récupérer de l'information; ils gèrent l'analyse.
            \end{itemize}
            Quand l'analyse est terminée, les structures présentes dans le Workspace représentent la compréhension du texte qu'a acquise l'analyseur.
            En en extrayant les plus importantes, on peut obtenir une synthèse du texte.
        \end{frame}



\end{document}

\documentclass{beamer}
\usepackage{pgfpages}
\usepackage[francais]{babel}
\usepackage[T1]{fontenc}
\usepackage[utf8]{inputenc}
\usepackage{amsmath,amsthm}
\usepackage{amsfonts,amssymb}

%%%%%%%%%%%%%%%%%%%%%%%%%%%%%%%%%%%%%%%%%%%%%%%%%%%%%%%%%%%%%%%%%%%%%%%%%%%%%%%
%THEMES DU DOCUMENT
%%%%%%%%%%%%%%%%%%%%%%%%%%%%%%%%%%%%%%%%%%%%%%%%%%%%%%%%%%%%%%%%%%%%%%%%%%%%%%%

%– th`emes globaux~: albatross, beetle, crane, fly, seagull.
%– th`emes internes~: lily  orchid, rose.
%– th`emes externes~: whale, seahorse, dolphin
%JuanLesPins Malmoe PaloAlto Berlin Boadilla Copenhagen Hannover Goettingen Montpellier Rochester
%Madrid Antibes Singapore Szeged Warsaw, thème par défaut Ilmenau Luebeck AnnArbor CambridgeUS Dresden

%themes de polices~: professionalfonts serif structurebold structureitalicserif structuresmallcapsserif 


\usetheme{Warsaw}
%\usecolortheme{dolphin}
\usefonttheme{structurebold}
%\useinnertheme{nom du theme interne}
%\useoutertheme{split}

%%%%%%%%%%%%%%%%%%%%%%%%%%%%%%%%%%%%%%%%%%%%%%%%%%%%%%%%%%%%%%%%%%%%%%%%%%%%%%%%%%%%%%%%
%AUTRES COMMANDES PRÉLIMINAIRES
%%%%%%%%%%%%%%%%%%%%%%%%%%%%%%%%%%%%%%%%%%%%%%%%%%%%%%%%%%%%%%%%%%%%%%%%%%%%%%%%%%%%%%%%%

\AtBeginSection[]{% génère une tableofcontents au début de chaque section
  \begin{frame}
  \frametitle{Sommaire}
	\tableofcontents[sectionstyle = show/shaded,subsectionstyle = show/show/shaded]
  \end{frame} 
}

\setbeamertemplate{itemize item}[triangle]
\setbeamertemplate{enumerate item}[square]
\setbeamertemplate{blocks} [rounded] [shadow=true]

\setbeameroption{show notes}%%ou hide notes
%%\setbeameroption{show notes on second screen=right}


\title{Réunion de lancement}
\subtitle{PSC INF02~: Synthétiseur automatique de documents}
\author{}
\institute{École polytechnique}
\date{}

\logo{\includegraphics[height=0.8cm]{logohori.eps}}

%%%%%%%%%%%%%%%%%%%%%%%%%%%%%%%%%%%%%%%%%%%%%%%%%%%%%%%%%%%%%%%%%%%%%%%%%%%%%%%%%%%
%DEBUT DU DOCUMENT
%%%%%%%%%%%%%%%%%%%%%%%%%%%%%%%%%%%%%%%%%%%%%%%%%%%%%%%%%%%%%%%%%%%%%%%%%%%%%%%%%%%

\begin{document}




%%page de titre
\begin{frame}
  \titlepage{}

\end{frame}		
%% sommaire %%		
\begin{frame}
\frametitle{Sommaire}
\tableofcontents
\end{frame}


%%%%%%%%%%%%%%%%%%%%%%%%%%%%%%%%%%%%%%%%%%%%%%%%%%%%%%%%%%%%%%%%%%%%%%%%%%%%
%VRAI DEBUT DU DOCUMENT
%%%%%%%%%%%%%%%%%%%%%%%%%%%%%%%%%%%%%%%%%%%%%%%%%%%%%%%%%%%%%%%%%%%%%%%%%%%%

\section{Introduction}

\begin{frame}%[allowframebreaks=0.8]
\frametitle{Objectifs du projet}
%%à l'ère du big data, une grande partie de ces données sont textuelles

\begin{block}{Des problèmes nombreux}
\begin{itemize}
 \item De grandes quantités de données textuelles
 \item Une grammaire complexe (langage naturel)
 \item Beaucoup de prérequis de bon sens ou de culture générale
\end{itemize}

\end{block}


\begin{block}{Plusieurs solutions envisageables}
\begin{itemize}
\item «~Comprendre~» l'information du texte et la retranscrire~: \textbf{résumé abstractif}
\item Analyser le texte de façon statistique et en extraire des phrases~: \textbf{résumé extractif}
\end{itemize}
\end{block}


\end{frame}


\begin{frame}
\frametitle{Notre projet~: Synthétiseur automatique de documents}

Nous voulons résumer un texte en langage naturel, de façon abstractive.

Le résumé final ne contiendra pas de phrases initialement présentes dans le texte.

\end{frame}

\section{Structure du projet}

\subsection{Le projet}
\begin{frame}
\frametitle{Le synthétiseur en six étapes}

\begin{columns}[t]

\begin{column}{6cm}
\begin{enumerate}
 \item Analyse syntaxique
% ie la grammaire. Nos exigences devraient entrer dans le champ des possibilités des outils libres déjà existants
 \item Repérage des entités.
 %les entités ne sont rien d'autre que des concepts importants
 \item Connaissances des liens entre entités.
 %liens qui sont donnés par FB~: Zlatan est un joueur de foot par exemple
 \item Analyse des concepts.
 %il s'agit de relier tous les mots à des concepts
 \item Approfondissement des liens entre les concepts.
 %explicité dans la suite
 \item Génération de texte simple
\end{enumerate}
\end{column}
%à droite, ce que ça donne aux différents stades

\begin{column}{6cm}
\begin{enumerate}
\item Sujet, Verbe, Objet\ldots{}
\item Zinedine Zidane\ldots{}
\item Zidane est un joueur de football\ldots{}
\item Le football est un sport\ldots{}
\item Zinedine Zidane est donc un sportif.
\item Sujets-Actions-Objets-Compléments
\end{enumerate}

\end{column}
\end{columns}

\end{frame}

\begin{frame}
\frametitle{Déroulement du projet}

\begin{enumerate}
\item Étude statistique
\item Étude syntaxique et linguistique
\item Étude «~intelligente~»
\item Génération et étude de la qualité du résumé
\end{enumerate}

\end{frame}

\subsection{Organisation du travail}

\begin{frame}
\frametitle{Méthodes de travail}
	\begin{itemize}
		\item Des réunions hebdomadaires, pour~:
		\begin{itemize}
			\item Faire le point sur le travail effectué dans la semaine;
			\item Rendre compte et résoudre les difficultés rencontrées;
			\item Définir les axes prioritaires pour la semaine à venir.
		\end{itemize}
		\item Un répertoire partagé pour une gestion pratique des fichiers.
	\end{itemize}
\end{frame}

\begin{frame}
\frametitle{Moyens utilisés}

De nombreux outils \textit{open-source} ou libres, notamment~:
\begin{itemize}
 \item Freebase
 \item WordNet (Princeton)
 \item nltk (librairie python)
\end{itemize}
\end{frame}



\section{Étude statistique}%Zhixing
%td-idf

\section{Réseau de concepts et structure associée}%Guillaume, change le titre
	

	
\end{document}
